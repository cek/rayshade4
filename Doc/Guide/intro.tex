\chapter{Introduction}
\pagenumbering{arabic}

This document describes {\rayshade} in enough detail to enable
the technical-minded to
sit down and render some images.  In its current form,
it is truly a draft, and even then
more of a reference manual than a proper user's guide.

This document does not provide any kind of thorough introduction
to the basics of computer graphics or ray tracing.
There are many other excellent sources for this kind of information.
The technical and coding details of {\rayshade} and
its libraries will be documented elsewhere.

\section{Getting Started}

The best way to learn how to use {\rayshade} is to dive right in and
start making pictures.  Study the example
input files that are packaged with {\rayshade}.  Run them through
{\rayshade} to see what the images they produce look like.
Change the input files; move the camera,
change the field of view, modify surface properties, and see what differences
your changes make, all the while referring to the appropriate portions
of this document.  Browse through the individual chapters
to see what {\rayshade} can and cannot do.  The {\rayshade} quick reference
guide may also help you sort out syntactical nasties.

Throughout this text, the {\tt typewriter} type style is used to indicate
keywords and other items that should be passed
directly to {\rayshade}.  Where appropriate,
items in an {\em italic\/} style indicate
places where you should provide an appropriate number or string.

Vectors, which consist of three numerical values, are indicated by
an arrow over a name written in italic type style, e.g., \evec{vector}.
Items enclosed between {\tt [} and {\tt ]} characters indicate
that specifying those items is optional.
Complex constructions that are
described elsewhere in the text, such as surface or object specification,
are denoted by enclosing descriptive text between 
{\tt $<$} and {\tt $>$} characters.

\section{A Simple Example}

Because {\rayshade} provides a default camera description, surface properties,
and a default light
source, it is easy to construct short input files that allow you to
experiment with objects, textures, and transformations.
If you haven't already run {\rayshade} on one of the example input files,
you might want to try producing an image using the following input:

\begin{verbatim}
   sphere 2 0 0 0
\end{verbatim}

If you are running {\rayshade} on a UNIX\footnote{UNIX is a trademark
of AT\&T Bell Laboratories}-like machine, the command:
\begin{verbatim}
  echo "sphere 2 0 0 0" | rayshade > sphere.rle
\end{verbatim}
should produce an image of a sphere.
