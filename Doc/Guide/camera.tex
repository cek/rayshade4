\chapter{Specifying a View}

When designing a {\rayshade} input file, there are two main
issues that must be considered.  The first and more complex
is the selection of the objects to be rendered and the appearances
they should be assigned.  The second and usually easier issue is
the choice of viewing parameters.  This chapter deals with the
latter problem; the majority of the following chapters
discuss aspects of objects and their appearances.

{\Rayshade} uses a camera model to describe
the geometric relationship between the objects to be rendered
and the image that is produced.  This relationship describes
a perspective projection from world space onto the image plane.

The geometry of the perspective projection
may be thought of as an
infinite pyramid, known as the viewing {\em frustum}.
The apex of the frustum is defined by the camera's position,
and the main axis of the frustum by a ``look'' vector.
The four sides of
the pyramid are differentiated by their relationship to a
reference ``up'' vector from the camera's position.

The image
ultimately produced by {\rayshade} may then be thought of as
the projection of the objects closest to the eye onto
a rectangular screen formed by the intersection of the pyramid with
a plane orthogonal to the pyramid's axis.  The overall shape
of the frustum (the lengths of the top and bottom sides compared
to left and right)
is described by the horizontal and vertical fields
of view.

\section{Camera Position}

The three basic camera properties are its position, the direction
in which it is pointing, and its orientation.  The keywords for
specifying these values are described below.
The default values
are designed to provide a reasonable view of a sphere of radius 2
located at origin.  If these default values are used,
the origin is projected onto the center of the image plane, with
the world $x$ axis  running left-to-right, the $z$ axis bottom-to-top,
and the $y$ axis going ``into'' the screen.

\begin{defkey}{eyep}{\evec{pos}}
	Place the virtual camera at the given position.
\end{defkey}
The default camera position is (0, -8, 0).

\begin{defkey}{lookp}{\evec{pos}}
	Point the virtual camera toward the given position.
\end{defkey}
The default look point is the origin (0, 0, 0).  The look point
and camera position must not be coincident.

\begin{defkey}{up}{\evec{direction}}
	The ``up'' vector from the camera point is set to the
	given direction.
\end{defkey}
This up vector need not be orthogonal to
the view vector, nor need it be normalized.  The default up
direction is (0, 0, 1).

Another popular standard viewing geometry, with the $x$ axis running
left-to-right,
the $y$ axis bottom-to-top, and the $z$ axis pointing out of the screen,
may be obtained by setting the up vector to (0, 1, 0) and by placing
the camera on the positive $z$ axis.

\section{Field of View}

Another important choice to be made is that of the 
field of view of the camera.  The size of this field describes
the angles between the left and right sides and top and bottom sides
of the frustum.

\begin{defkey}{fov}{{\em hfov} [{\em vfov}]}
	Specify
	the horizontal and vertical field of view, in degrees.
\end{defkey}
The default horizontal field of view is 45 degrees.
If {\em vfov} is omitted, as is the general practice,
the vertical field of view is computed using the horizontal
field of view, the output image resolution, and the assumption
that a pixel samples a square area.  Thus,
the values passed via the
{\tt screen} keyword define the shape of the final image.
If you are
displaying on a non-square pixeled device,
you must set the vertical field of view
to compensate for the ``squashing'' that will result.

\section{Depth of Field}

Under many circumstances,
it is desirable to render
objects in the image such that they are in sharp
focus on the image plane.  This is achieved by using the default
``pinhole' camera.  In this mode, the camera's aperture is a single
point, and all light rays are focused on the image plane.

Alternatively, one may widen the aperture in order to
simulate depth of field. In this case, rays are cast from various places on
the aperture disk towards a point whose distance from the
camera is equal to the
{\em focus distance}.  Objects that lay in the focal plane will be
in sharp focus.  The farther an object is from the image plane,
the more out-of-focus it will appear to be.
A wider aperture will lead to a greater blurring of
objects that do not lay in the focal plane.
When using a non-zero aperture radius, it is best to use jittered
sampling in order to reduce aliasing.

\begin{defkey}{aperture}{{\em radius}}
	Use an aperture with the given {\em radius}.
\end{defkey}
The default radius is zero, resulting in a pinhole camera model.

\begin{defkey}{focaldist}{{\em distance}}
	Set the focal plane to be {\em distance} units from the camera.
\end{defkey}
By default, the focal distance is equal to the distance from the
camera to the look point.

\section{Stereo Rendering}

Producing a stereo pair is a relatively simple process; rather than
simply rendering a single image, one creates two related images which
may then be viewed on a stereo monitor, in a stereo slide viewer, or
by using colored glasses and an appropriate display filter.

{\Rayshade} facilitates the rendering of stereo pairs by allowing you
to specify the distance between the camera positions used in
creating the two images.  The camera position given in the
{\rayshade} input file
defines
the midpoint between the two camera positions used to generate the
images.
Generally, the remainder of the viewing parameters are kept constant.

\begin{defkey}{eyesep}{{\em separation}}
	Specifies the camera separation to be used in rendering stereo
	pairs.
\end{defkey}
There is no default value.
The separation may also be specified on the command line through
the {\em -E} option.
The view to be rendered (left or right)
must be specified on the command line by using
the {\tt -l} or {\tt -r} options.

There are several things to keep in mind when generating stereo
pairs.  Firstly, those objects that lie in from of the focal plane will
appear to protrude from the screen when viewed in stereo, while objects
farther
than the focal plane will recede into the screen.  As it is usually
easier to look at stereo images that recede into the screen, you will
usually
want to place the look point closer to the camera than the object
of primary interest.

The degree of stereo effect is a
function of the camera separation and the distance from the camera
to the look point. Too large a separation will result in a hyperstereo
effect that will be hard to resolve, while too little a value will result
in no stereo effect
at all.  A separation equal to one tenth the distance from the
camera to the look point is often a good choice.
