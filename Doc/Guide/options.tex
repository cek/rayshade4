\chapter{Options}

This appendix describes the command-line arguments accepted by {\rayshade}.
These options override defaults
as well as any values or flags given in the input file,
and are thus useful for generating test and other unusual, ``non-standard''
renderings.

The general form of a {\rayshade} command line is:
\begin{quote}
{\tt rayshade} [{\em Options}] [{\em filename}]
\end{quote}

If given, the input file is read from {\em filename}.  By default,
the input file is read from the standard input.
Recall that, by default, the image file is written to the standard
output; you will need to redirect the standard output if you have not
chosen to write the image to a file directly.  The name of the input
file may be given anywhere on the command line.

Command-line options fall into two broad categories: those that set
numerical or other values and thus must be followed by further arguments,
and those that simply turn features on and off.  {\Rayshade}'s
convention is to denote the value-setting arguments using capital letters,
and feature-toggling arguments using lower-case letters.

\begin{defkey}{-A}{{\em frame}}
	Begin rendering (action) on the given frame.
\end{defkey}
The default starting frame is number zero.

\begin{defkey}{-a}{}
	Toggle writing of alpha channel.
\end{defkey}
This option is only available when the Utah Raster Toolkit is
being used.

\begin{defkey}{-C}{{\em R G B}}
	Set the adaptive ray tree pruning color.  If all
	channel contributions falls below the given cutoff
	values, no further rays are spawned.
\end{defkey}
Overrides the value specified via the {\tt cutoff} keyword.

\begin{defkey}{-c}{}
	Continue an interrupted rendering.
\end{defkey}
When given, this option indicates that the image file being written
to contains a partially-completed image.  {\Rayshade} will read the
image to determine the scanline from which to continue the rendering.
This option is only available with the Utah Raster Toolkit.
The {\tt -O} option must also be used.

\begin{defkey}{-D}{{\em depth}}
	Set maximum ray tree depth.
\end{defkey}
Overrides the value specified in the input file via the {\tt maxdepth}
keyword.

\begin{defkey}{-E}{{\em separation}}
	Set eye separation for rendering of stereo pairs.
\end{defkey}
Overrides the value specified via the {\tt eyesep} keyword.

\begin{defkey}{-e}{}
	Write exponential RLE file.
\end{defkey}
This option is only available for use with the Utah Raster Toolkit.
See the Utah Raster Toolkit's {\em unexp} manual page for details on
exponential RLE files.

\begin{defkey}{-F}{{\em freq}}
	Set frequency of status report.
\end{defkey}
Overrides the value given using the {\tt report} keyword.

\begin{defkey}{-f}{}
	Flip all computed polygon (and triangle) normals.
\end{defkey}
This option should be used when rendering polygons defined
by vertices given in {\em clockwise}
order, rather than counter-clockwise order as
expected by {\rayshade}.

\begin{defkey}{-G}{{\em gamma}}
	Use given gamma correction exponent writing writing
	color information to the image file.
\end{defkey}
The default value for {\em gamma} is 1.0.

\begin{defkey}{-g}{}
	Use a Gaussian pixel filter.
\end{defkey}
Overrides the filter selected through the use of the {\tt filter}
keyword.

\begin{defkey}{-h}{}
	Print a short use message.
\end{defkey}

\begin{defkey}{-j}{}
	Toggle the use of jittered sampling to perform antialiasing.
	If disabled, a fixed sampling pattern is used.
\end{defkey}

\begin{defkey}{-l}{}
	Render the left stereo pair image.
\end{defkey}

\begin{defkey}{-m}{}
	Write a sampling map to the alpha channel.
\end{defkey}
Rather than containing coverage information, the alpha channel values
will be restricted to zero, indicating no supersampling, and full intensity,
indicating supersampling.  This option is only available if the Utah
Raster Toolkit is being used.

\begin{defkey}{-N}{{\em frames}}
	Set the total number of frames to be rendered.
\end{defkey}
This option overrides any value specified through the use of the
{\tt frames} keyword.  By default, a single frame is rendered.

\begin{defkey}{-n}{}
	Do not render shadows.
\end{defkey}

\begin{defkey}{-O}{{\em outfile}}
	Write the image to the named file.
\end{defkey}
This option overrides the name given with the {\tt outfile} keyword,
if any,
in the input file.

\begin{defkey}{-o}{}
	Toggle the effect of object opacity on shadows.
\end{defkey}
This option is equivalent to specifying {\tt shadowtransp}
in the input file.  By default, {\rayshade} traces shadow
rays through non-opaque objects.

\begin{defkey}{-P}{{\em cpp-arguments}}
	Specify the options that should be passed to the C
	preprocessor.
\end{defkey}
The C preprocessor, if available, is applied to all of the input
passed to {\em rayshade}.

\begin{defkey}{-p}{}
	Perform preview-quality rendering.
\end{defkey}
This option is equivalent to {\tt -n -S 1 -D 0}.

\begin{defkey}{-q}{}
	Do not print warning messages.
\end{defkey}

\begin{defkey}{-R}{{\em xsize ysize}}
	Produce an image {\em xsize} pixels wide by
	{\em ysize} pixels high.
\end{defkey}
This option overrides any screen size set by use of
the {\tt screen} keyword.

\begin{defkey}{-r}{}
	Render the right stereo pair image.
\end{defkey}

\begin{defkey}{-S}{{\em samples}}
	Use $samples^2$ jittered samples.
\end{defkey}
This option overrides any value set through the use of
the {\tt samples} keyword in the input file.

\begin{defkey}{-s}{}
	Disable caching of  shadowing information.
\end{defkey}
It should never be necessary to use this option.

\begin{defkey}{-T}{{\em r g b}}
	Set the contrast threshold in the three
	color channels for use in adaptive supersampling.
\end{defkey}
This option overrides any value given through the use of
the {\em contrast} keyword.

\begin{defkey}{-u}{}
	Toggle the use of the C preprocessor.
\end{defkey}
{\Rayshade} usually feeds its input through a
C preprocessor if one is available on your system.
If this option is given, unadulterated input files will
be used.

\begin{defkey}{-V}{{\em	filename}}
	Write verbose output to the named file.
\end{defkey}
This option overrides any file named through the use of
the {\tt report} keyword.

\begin{defkey}{-v}{}
	Write verbose output.
\end{defkey}
When given, this option causes information about the options
selected and the objects defined to be included in the
report file.

\begin{defkey}{-W}{{\em minx maxx miny maxy}}
	Render the specified window.
\end{defkey}
The window must be properly contained within the screen.  This
option overrides any window specified using the {\em window} keyword
in the input file.

\begin{defkey}{-X}{{\em left right bottom top}}
	Crop the rendering window using the given normalized values.
\end{defkey}
This option is provided to facilitate changing and/or examining a
small portion of an image without having to re-render the entire
image.
